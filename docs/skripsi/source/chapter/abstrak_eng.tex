\chapter*{\textbf{ABSTRACT}}

\textbf{FARHAN DEWANTA SYAHPUTRA}. Perancangan dan Implementasi \emph{Prototype Database Engine} Berbasis 
\emph{Structure Oriented Programming} Menggunakan Rust. Mini Thesis. Computer Science. Faculty of Mathematics
and Natural Sciences, State University of Jakarta. July 2025. Under guidance from
Muhammad Eka Suryana, M.Kom and Med Irzal, M.Kom. 


\vspace{5mm}
\noindent{}
Database are becoming an important system in this era. Every database have they own license, starting from strict license 
to permissive one. For strictly license database, developers cannot have full control to the entire source code itself.
Meanwhile, some permissive license database also have a complex code structure which will be difficult to customized based on requirements.
Therefore, this research about making database engine are made for the purposes of developers so they can implement
their own logic and algorithm based by their own needs, starting from simple layer codebase. For this development case, 
synchronizationn algorithm for distributed database can be implemented in the future. Database engine feature will be developed 
based by basic feature of existing database so it will fulfill the industry needs. The result of this database engine research are the entire process of reading, 
creating, updating and deleting can be seen. Developers also can make their own configuration based on their needs. Even in the 
final result of this reasearch, there are still much things to fix and develop to make these database better.

\vspace{5mm}
\noindent{}
\textbf{Keyword}: \emph{database engine, index, DBMS, Distributed System}