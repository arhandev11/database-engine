\chapter*{\textbf{ABSTRAK}}

\textbf{FARHAN DEWANTA SYAHPUTRA}. Perancangan dan Implementasi \emph{Prototype Database Engine} Berbasis 
\emph{Structure Oriented Programming} Menggunakan Rust. Skripsi. Program Studi Ilmu Komputer. Fakultas Matematika
dan Ilmu Pengetahuan Alam, Universitas Negeri Jakarta. Juli 2025. Di bawah bimbingan
Muhammad Eka Suryana, M.Kom dan Med Irzal, M.Kom. 


\vspace{5mm}
\noindent{}
\emph{Database} menjadi sistem yang sangat penting untuk digunakan di era saat ini.
Setiap \emph{database} yang ada memiliki lisensi yang berbeda-beda, mulai dari yang ketat hingga
\emph{permissive}. Untuk \emph{database} yang memiliki lisensi ketat, maka pengembang tidak dapat
mengendalikan \emph{source code} secara utuh. Sementara, sebagian \emph{database} yang memiliki lisensi \emph{permissive}
juga sudah memiliki struktur \emph{code} yang kompleks sehingga akan sulit untuk dikembangkan sesuai dengan kebutuhan.
Maka dari itu penelitian pembuatan \emph{database engine} dilakukan dengan tujuan kedepannya para pengembang dapat 
mengembangkan algoritma dan sistem penyimpanan sesuai dengan yang dibutuhkan, mulai dari lapisan yang sederhana. 
Pada kasus ini pengembangan ini, kedepannya algoritma sinkronisasi untuk sistem terdistribusi dapat diterapkan 
pada \emph{database engine}. Pengembangan fitur \emph{Database Engine} akan dilakukan berdasarkan fitur-fitur dasar \emph{database} yang umum untuk digunakan
agar dapat memenuhi kebutuhan industri saat ini. Hasil akhir dari penelitian mengenai \emph{database engine} ini adalah 
semua pemrosesan dari menampilkan, membuat, mengubah, dan menghapus dapat dilihat dengan jelas. Pengembang
juga bisa melakukan pengaturan sesuai dengan kebutuhannya. Meski pada hasil akhir penelitian,
masih terdapat beberapa fitur yang harus diperbaiki dan disempurnakan.

\vspace{5mm}
\noindent{}
\textbf{Kata kunci}: \emph{database engine, index, DBMS, sistem terdistribusi}
