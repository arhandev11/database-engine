%!TEX root = ../main.tex
%-------------------------------------------------------------------------------
%                          BAB V
%               		KESIMPULAN DAN SARAN
%-------------------------------------------------------------------------------

\chapter{KESIMPULAN DAN SARAN}

\section{Kesimpulan}

Berdasarkan hasil implementasi dan pengujian yang sudah dijalankan maka terdapat beberapa kesimpulan sebagai berikut:

\begin{enumerate}
	\item Untuk membuat \emph{database engine} yang menjaga integritas data, maka harus memikirkan bagaimana cara
	database tersebut disimpan pada \emph{filesystem}. Jenis data yang disimpan juga harus dipikirkan untuk memastikan
	data yang telah berubah menjadi \emph{byte} dapat kembali menjadi nilai asli dan tipe aslinya walaupun pada fitur saat ini 
	masih memiliki keterbatasan dalam pengembalian datanya.
	\item Penyimpanan data pada \emph{filesystem} harus konsisten agar tidak terdapat data yang hilang. Karena berbeda 1 urutan saja
	dapat menghancurkan data-data yang tersimpan diurutan lainnya.
	\item Fitur-fitur untuk mengolah data yang disimpan pada \emph{filesystem} dapat dikelola oleh atribut dan \emph{method}
	pada bahasa pemrograman yang digunakan. Sehingga harus mempertimbangkan \emph{method} yang mana saja yang dapat di akses oleh \emph{client}.
	\item Tipe data yang terdapat pada \emph{database} harus kompatibel untuk berbagai bahasa untuk memastikan antara server dan\emph{client}dapat saling
	tukar menukar data.
	\item Implementasi algoritma sinkronisasi sangat memungkinkan untuk diterapkan di dalam sistem \emph{database} karena alur awal penerimaan data sampai
	penyimpanan data dapat dilihat secara langsung. Tak hanya algoritma, metode penyimpanan pada \emph{filesystem} dan \emph{index} juga bisa lebih dikembangkan lagi
	untuk mencari performa yang paling optimal.
	\item Penanganan penyimpanan data untuk bahasa non latin memiliki tahapan yang berbeda dikarenakan cara yang diterapkan pada saat ini
	masih belum berhasil.
\end{enumerate}

\section{Saran}

Berdasarkan hasil implementasi dan pengujian yang sudah dijalankan, terdapat beberapa saran untuk menyempurnakan \emph{database engine} di kemudian hari, yaitu:

\begin{enumerate}
	\item Mencoba memperbaiki kekurangan pada \emph{database engine} sesuai dengan yang dibahas pada sub bab 4.2.4. Tujuannya adalah agar \emph{database engine} yang telah dibuat 
	saat ini dapat diimplementasikan pada sistem lain yang sudah berjalan.
	\item Mengimplementasikan fitur distribusi \emph{database}, agar dapat segera digunakan pada sistem \emph{crawling} untuk \emph{search engine}. Salah satu langkah utamanya
	adalah dengan mencari algoritma sinkronisasi yang paling optimal dan efisien yang dapat diterapkan pada \emph{database engine} saat ini.
	\item Implementasikan lebih dalam fitur-fitur koneksi yang ada pada D-Bus agar komunikasi ke \emph{client} dapat berjalan lebih baik.
	\item Sebelum melanjutkan pengembangan \emph{database}, sangat disarankan untuk mempelajari konsep dan paradigma yang ada di bahasa pemrograman rust agar dapat mengimplementasikan
	fitur-fitur lain menjadi lebih baik ke depannya.
\end{enumerate}


% Baris ini digunakan untuk membantu dalam melakukan sitasi
% Karena diapit dengan comment, maka baris ini akan diabaikan
% oleh compiler LaTeX.
\begin{comment}
\bibliography{daftar-pustaka}
\end{comment}
