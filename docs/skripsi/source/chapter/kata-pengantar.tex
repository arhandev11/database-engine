\chapter*{\centering{\large{KATA PENGANTAR}}}
\onehalfspacing{}
Segala puji bagi Allah Subhanahu wa ta'ala Rabb semesta alam. Rasa syukur penulis 
panjatkan kehadirat Allah Subahanahu wa ta'ala, karena berkat rahmat, hidayah, 
taufiq serta karunia-Nya penulis dapat menyelesaikan skripsi
yang berjudul Perancangan dan Implementasi \emph{Prototype Database Engine} Berbasis 
\emph{Structure Oriented Programming} Menggunakan Rust. 

Selama penulisan skripsi ini, banyak pihak yang memberikan bantuan dan dukungan 
untuk penulis agar penulis senantiasa menyelesaikan skripsi ini. Bantuan yang diberikan
berupa bantuan secara langsung dan juga moral untuk penulis. Melalui kesempatan ini, 
penulis ingin berterima kasih kepada orang-orang tersebut, yaitu:

\begin{enumerate}
	\item{Yth. Para petinggi di lingkungan FMIPA Universitas Negeri Jakarta.}
	\item{Yth. Ibu Dr. Ria Arafiyah, M.Si selaku Koordinator Program Studi Ilmu
		Komputer.}
	\item{Yth. Bapak Muhammad Eka Suryana, M.Kom selaku Dosen Pembimbing I, dengan 
	sabar telah membimbing, membantu, mengarahkan, dan mengoreksi selama proses 
	penulisan skripsi ini berlangsung}
	\item{Yth. Bapak Med Irzal, M.Kom selaku Dosen Pembimbing II dengan 
	sabar telah membimbing, membantu, mengarahkan, dan mengoreksi selama proses 
	penulisan skripsi ini berlangsung}
	\item{Kedua orang tua, abang dan kakak penulis yang senantiasa memberikan 
		dukungan moral serta doa untuk penulis.}
	\item{Teman-teman Warung Tegang yang tidak pernah berhenti mendukung penulis selama proses penulisan skripsi}
	\item{Teman-teman PT Amanah Karya Indonesia yang sering memberikan keringanan dalam bekerja ketika penulis menulis skripsi}
	\item{Teman-teman Program Studi Ilmu Komputer Universita Negeri Jakarta yang telah memberikan 
		dukungan moral dalam penulisan skripsi ini.}
\end{enumerate}

Penulis mengakui bahwa skripsi ini masih sangat jauh dari kata sempurna sehingga penulis 
sangat terbuka terhadap saran dan kritikan yang membangun. Semoga segala hal yang telah 
penulis buat dalam skripsi ini akan bermanfaat bagi orang lain, karena sebagaimana sabda 
Rasulullah Sholallahu 'alaihi wa salam bahwa sebaik-baiknya manusia adalah yang paling 
bermanfaat bagi manusia. Semoga Allah Subhanahu wa ta'ala membalas segala kebaikan yang telah diberikan 
kepada orang-orang yang telah membantu penulis.

\vspace{4cm}

\begin{tabular}{p{7.5cm}c}
	&Jakarta, 12 Agustus 2025\\
	&\\
	&\\
	&\\
	&Farhan Dewanta Syahputra
\end{tabular}
